%\title{Title page with logo}
%----------------------------------------------------------------------------------------
%	PACKAGES AND OTHER DOCUMENT CONFIGURATIONS
%----------------------------------------------------------------------------------------

\documentclass[12pt]{article}
\usepackage[portuges]{babel}
\usepackage[utf8]{inputenc}
\usepackage[hyphens]{url}
\usepackage{graphicx}
\usepackage[hidelinks]{hyperref}
\usepackage[toc,page]{appendix}
\usepackage{indentfirst}
\usepackage[table,xcdraw]{xcolor}
\usepackage{longtable}
\usepackage[margin=4cm]{geometry}

\newcommand{\tabitem}{~~\llap{\textbullet}~~}

\renewcommand\appendixtocname{Anexos}
\renewcommand\appendixpagename{Anexos}

\begin{document}
\sloppy
\LTcapwidth=\textwidth

\begin{titlepage}

\newcommand{\HRule}{\rule{\linewidth}{0.5mm}} % Defines a new command for the horizontal lines, change thickness here

\center % Center everything on the page

\vspace{0.5cm}
 
%----------------------------------------------------------------------------------------
%	HEADING SECTIONS
%----------------------------------------------------------------------------------------

\textsc{\LARGE Scripting no Processamento}\\[0.3cm]
\textsc{\LARGE de Linguagem Natural}\\[1.1cm] % Name of your university/college
\textsc{\Large Universidade do Minho}\\[0.5cm] % Major heading such as course name
\textsc{\large Mestrado Integrado em Engenharia Informática}\\[0.5cm] % Minor heading such as course title

%----------------------------------------------------------------------------------------
%	TITLE SECTION
%----------------------------------------------------------------------------------------
\vspace{0.8cm}
\HRule \\[0.6cm]
{ \huge \bfseries TP2}\\[0.4cm] % Title of your document
{ \Large \bfseries Spacy's POS Tagging}\\[0.4cm] % Subtitle of your document
\HRule \\[1.0cm]
 
%----------------------------------------------------------------------------------------
%	AUTHOR SECTION
%----------------------------------------------------------------------------------------

\Large \emph{Grupo 7:}\\
A73831 - João Miguel Pires Barreira\\
A77364 - Mafalda Nunes\\[1.3cm]

%----------------------------------------------------------------------------------------
%	DATE SECTION
%----------------------------------------------------------------------------------------

{\large \today}\\[1.5cm] % Date, change the \today to a set date if you want to be precise

%----------------------------------------------------------------------------------------
%	LOGO SECTION
%----------------------------------------------------------------------------------------

\includegraphics[width=0.55\textwidth]{logo}\\[1cm] % Include a department/university logo - this will require the graphicx package
 
%----------------------------------------------------------------------------------------

\vfill % Fill the rest of the page with whitespace

\end{titlepage}

\vspace{0.5cm}

\begin{abstract}
O presente relatório tem com objetivo a aprendizagem das funcionalidades da ferramenta \textit{Spacy}. Para tal, apresentar-se-á uma descrição da mesma, mais especificamente da funcionalidade de POS \textit{tagging}, bem como um pequeno exemplo que demonstre como utilizar a ferramenta.

Este trabalho pretende dar resposta ao trabalho prático 2, proposto na unidade curricular SPLN, do Mestrado em Engenharia Informática, da Universidade do Minho.
\end{abstract}

\vspace{0.5cm}

\tableofcontents

\newpage

\let\oldref\ref
\renewcommand{\ref}[1]{\smash{\underline{\oldref{#1}}}}

\section{Spacy's}

% Falar sobre o que é
% https://spacy.io/usage/spacy-101
% https://spacy.io/usage/v2

\subsection{Arquitetura}

% https://spacy.io/api/

\subsection{Funcionalidades}

% Falar de forma relativamente geral, mas com imagens para se perceber (informação de POS tagging pode ser aprofundada e utilizada na próxima secção)

% https://spacy.io/usage/spacy-101
% https://spacy.io/usage/linguistic-features
% https://spacy.io/usage/processing-pipelines
% https://spacy.io/usage/vectors-similarity
% https://spacy.io/usage/training
% https://spacy.io/usage/adding-languages
% https://spacy.io/usage/visualizers

% https://spacy.io/usage/v2
% https://spacy.io/api/annotation
% https://spacy.io/api/top-level

% https://towardsdatascience.com/a-short-introduction-to-nlp-in-python-with-spacy-d0aa819af3ad
% https://towardsdatascience.com/a-review-of-named-entity-recognition-ner-using-automatic-summarization-of-resumes-5248a75de175


\subsection{Vantagens}

% https://spacy.io/usage/facts-figures
% https://www.analyticsvidhya.com/blog/2017/04/natural-language-processing-made-easy-using-spacy-%E2%80%8Bin-python/
% https://medium.com/@brianray_7981/ai-in-practice-identifying-parts-of-speech-in-python-8a690c7a1a08

\subsection{Utilização}

% https://spacy.io/api/cli
% https://spacy.io/usage/



\subsection{Aplicações}

% https://spacy.io/usage/examples
% https://github.com/explosion/spacy/blob/master/examples/information_extraction/entity_relations.py


% Implementar?
% https://books.google.pt/books?id=48RiDwAAQBAJ&pg=PA80&lpg=PA80&dq=spacy+attrs&source=bl&ots=R2y5H5s6e5&sig=33zo94rAGOHBR5PxOH51J1GrBGM&hl=pt-PT&sa=X&ved=2ahUKEwiNlpbKoO_eAhWmJcAKHetlCfQ4ChDoATAAegQICRAB#v=onepage&q=spacy%20attrs&f=false


\section{Spacy's POS Tagging}

% Aprofundar tema usando websites relevantes indicados nas funcionalidades


\subsection{Utilização}


\section{Exemplo Demonstrativo}

% Falar das funções que desenvolvemos





% Anexos

\setcounter{section}{0}
\setcounter{subsection}{0}


\newpage

\appendixpage
\renewcommand{\thesubsection}{\Alph{subsection}}

% Questão 1

%\subsection{Título}
%\label{anexo:nome}
%\begin{figure}[!ht]
%	\centering
%	\makebox[\textwidth][c]{\includegraphics[width=13cm]{Pictures/...}}
%	\caption{Legenda}
%\end{figure}

\newpage

\begin{thebibliography}{99}
	
	% Questão 1
	
	\bibitem{linkname}
	Autor,
	``Título'',
	\textit{website name},
	\url{website link}.
	
\end{thebibliography}

\end{document}